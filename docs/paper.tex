\documentclass{article}

\title{RASTiC: Random Autoregressive Subsampling for Token Integrity Checking and its Implementation}
\author{Dylan Dunn}
\date{\today}

\begin{document}

\maketitle

\begin{abstract}
This paper presents a novel method for verifying the integrity of autoregressive language models based on random subsampling of dependent tokens. The method, called RASTiC, leverages the auto-regressive nature of these models and samples a subset of tokens in a text sequence to check that the preceding tokens can generate the subsequent one. We describe the implementation of RASTiC and evaluate its performance on several large language models. Our experiments show that RASTiC can detect generated text with a high level of accuracy and is more efficient than traditional spot-checking methods. We also discuss the potential limitations and possible extensions of the method, including its vulnerability to targeted attacks. Overall, we believe that RASTiC provides a simple yet effective way to improve the trustworthiness of autoregressive language models and can be easily incorporated into existing frameworks.
\end{abstract}


\section{Introduction}
The introduction sets the context for your research paper by providing background information, stating the problem or research question, and explaining the significance of your study. It should also provide an overview of your methodology and briefly state your findings.

\section{Methodology}
The methodology section describes the methods you used to conduct your study, including the data sources, sampling methods, and data analysis techniques. It should also describe how you implemented RASTiC in your study.

\section{Results}
The results section presents the findings of your study, including any statistical analyses or other data visualizations. It should also include a discussion of your findings, including any limitations or implications.

\section{Discussion}
The discussion section provides a more detailed analysis of your findings, including any insights or interpretations that you have drawn from them. It should also include a discussion of the broader implications of your study, including any policy or practice recommendations.

\section{Conclusion}
The conclusion summarizes the main findings of your study and restates its significance. It should also suggest directions for future research.

\end{document}

You can customize this template to fit the requirements of your specific research paper. You can add additional sections or subsections as needed, and you can use LaTeX commands to format your text, add figures and tables, and include citations and references.
